%%%%%%%%%%%%%%%%%%%%%%%%%%%%%%%%%%%%%%%%%
% Awesome Cover Letter
% XeLaTeX Template
% Version 1.1 (9/1/2016)
%
% This template has been downloaded from:
% http://www.LaTeXTemplates.com
%
% Original authors:
% Claud D. Park (posquit0.bj@gmail.com)
% Lars Richter (mail@ayeks.de)
% With modifications by:
% Vel (vel@latextemplates.com)
%
% License:
% CC BY-NC-SA 3.0 (http://creativecommons.org/licenses/by-nc-sa/3.0/)
%
% Important note:
% This template must be compiled with XeLaTeX, the below lines will ensure this
%!TEX TS-program = xelatex
%!TEX encoding = UTF-8 Unicode
%
%%%%%%%%%%%%%%%%%%%%%%%%%%%%%%%%%%%%%%%%%

%----------------------------------------------------------------------------------------
%	PACKAGES AND OTHER DOCUMENT CONFIGURATIONS
%----------------------------------------------------------------------------------------

\documentclass[12pt, a4paper]{awesome-cv} % A4 paper size by default, use 'letterpaper' for US letter

\geometry{left=2cm, top=0.8cm, right=2cm, bottom=2cm, footskip=.5cm} % Configure page margins with geometry
 
\fontdir[fonts/] % Specify the location of the included fonts

% Color for highlights
\colorlet{awesome}{awesome-red} % Default colors include: awesome-emerald, awesome-skyblue, awesome-red, awesome-pink, awesome-orange, awesome-nephritis, awesome-concrete, awesome-darknight
%\definecolor{awesome}{HTML}{CA63A8} % Uncomment if you would like to specify your own color

% Colors for text - uncomment and modify
%\definecolor{darktext}{HTML}{000000}
%\definecolor{text}{HTML}{000000}
\definecolor{graytext}{HTML}{101111}
%\definecolor{lighttext}{HTML}{000000}

\renewcommand{\acvHeaderSocialSep}{\quad\textbar\quad} % If you would like to change the social information separator from a pipe (|) to something else

%----------------------------------------------------------------------------------------
%	PERSONAL INFORMATION
%	Comment any of the lines below if they are not required
%----------------------------------------------------------------------------------------

\name{Benjamin Henry}{Pallay}
\address{14 McCord Street,Gordon Park,Brisbane,4031}
\mobile{(04) 5625-9095}

\email{benjaminpallay@gmail.com}
%\homepage{www.posquit0.com}
\github{benjaminpallay}
\linkedin{ben-pallay}
%\skype{skypeid}
%\stackoverflow{SOid}{SOname}
%\twitter{@twit}

%----------------------------------------------------------------------------------------
%	RECIPIENT/POSITION/LETTER INFORMATION
%	All of the below lines must be filled out
%----------------------------------------------------------------------------------------

\recipient{Hiring Manager, ThoughtWorks}{} % The company being applied to

\letterdate{\today} % The date on the letter, default is the date of compilation

\lettertitle{Application for Graduate Software Developer} % The title of the letter

\letteropening{Dear Hiring Manager,} % How the letter is opened

\letterclosing{
	I have included my resume for your consideration and would appreciate the opportunity to discuss my application with you by phone, email, or at an interview.} % How the letter is closed

%\letterenclosure[Attached]{Curriculum Vitae} % Any enclosures with the letter

%\makecvfooter{\today}{Claud D. Park~~~·~~~Cover Letter}{} % Specify the letter footer with 3 arguments: (<left>, <center>, <right>), leave any of these blank if they are not needed
  
%----------------------------------------------------------------------------------------

\begin{document}

\makecvheader % Print the header

\makelettertitle % Print the title

%----------------------------------------------------------------------------------------
%	LETTER CONTENT
%----------------------------------------------------------------------------------------

\begin{cvletter}

%------------------------------------------------

I am a recent graduate from the University of Queensland having completed my Bachelor of Science with majors in Mathematics and Computer Science. I'm interested in the increasing adoption of technology throughout our society,the effects it has on our lives and generally just technology in itself.

I completed quite a few interesting projects throughout my degree. There are two major projects that I want to describe briefly. The first was a project in the field of artificial intelligence, involving guiding a simplified model of a wheelchair through various environments of varying complexity. The project was completed in Java. We were taught two different methods to solve similar problems. The method that I used, Rapidly Exploring Random Trees (RRT), was covered very briefly in the lecture, actually just half a page of lecture notes. This meant that I needed to do significant research to be able to solve the problem, so I read through many  research papers on the topic. I applied the knowledge that I gained from the research into my final project and received a distinction.

The second project was for the final course in my Computer Science degree. I took on the role of backend developer. Our group of six was tasked with creating a weather and travel recommendation application for a client. My role involved designing and developing an API for the frontend developers. I completed the API using Flask, a Python framework. To provide my frontend developers with data, I interacted with a MySQL database and Queensland Government SILO climate data sets. I also developed a token authorization system to ensure security of the application. This project helped develop my communication and collaboration skills significantly, especially relevant to working on a project as part of a team. Our client was very happy with our final product and my group received a distinction.

Outside of my university education, I've learnt quite a lot through my own study. One particular interest of mine has been in developing my functional programming skills. I'm quite comfortable using Clojure, having developed an application to scrape Reddit subreddits to download images from external websites. I've also been interested in automating system administration tasks. My proudest creation is a system using a combination of Bash and Python to automatically download and rename files from an external server to my media server.

Last year I became a Youth Leader at my church and joined the church band playing piano and keyboard. My role as Youth Leader involves leading Friday night activities and leading devotions and educational activities. Through my experience in these roles I've developed teamwork, collaboration and leadership skills that are essential in any role. I taught myself piano by ear and am now very proficient in improvisation. My interest was piqued partly because of the similarities between  patterns in music and in mathematics.

I am also very experienced in living, working and communicating cross-culturally, as I have both Australian and Solomon Island roots. I was born in Solomon Islands, although my education was fully Australian-based. My project teams at university were also culturally diverse. My awareness and skill in dealing with different cultural norms and expectations, language barriers and strong diverse work teams helps me to be a good fit within ThoughtWorks' strong culture of diversity.  

I believe this opportunity will give me a chance to apply the skills that I've learnt as well as learn useful new skills which will help in future roles. I believe I am a good candidate for the role and that I have a lot to offer ThoughtWorks. I would be ready to start in January 2018 but am willing to start at any time.

%------------------------------------------------

\end{cvletter}

%----------------------------------------------------------------------------------------

\makeletterclosing % Print the signature and enclosures

\end{document}